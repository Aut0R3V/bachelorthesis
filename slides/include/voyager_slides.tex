hatten\begin{frame2
    \frametitle{Voyager - Planung}
    \begin{itemize}
        \item Erste Sonde zur Erkundung der äußereren Planeten
        \item Basierend auf dem Termoelectric Outer Planets Spacecraft
        \item Ursprünglich 4-5 Sonden geplant
        \item Auf Lebensdauer von 4 Jahren ausgelegt
    \end{itemize}
\end{frame}


\begin{frame}
    \frametitle{Voyager - Vor dem Start}
    \begin{itemize}
        \item Projektstart 1972
        \item Starke Einschräkungen der NASA
        \item Uranus/Neptun nicht mehr Missionsziele
        \item Planung bis 1975 abgeschlossen
        \item Über 10000 mögliche Flugbahnen
    \end{itemize}
\end{frame}


\begin{frame}
    \frametitle{Aufbau der Sonden}
    \begin{itemize}
        \begin{columns}[c]
            \begin{column}[T]{5cm}
                \item Sonden fast identisch
                \item Vieles von Viking Sonden \\ kopiert
                \item Stromversorgung durch \\ Radioisotopenbatterien
                \item 16 Düsen für Kurskorrekturen
            \end{column}
            \begin{column}[T]{5.5cm}
    \begin{figure}
		        \includegraphics[height=6cm,width=5cm]{pics/voyager_sonde.png}
                \caption{Zeichnung der Sonde\cite{Leitenberger_Voyager}}
                \end{figure}
            \end{column}
        \end{columns}
    \end{itemize}
\end{frame}


\begin{frame}
    \frametitle{Boardcomputer}
    \begin{itemize}
        \item Einiges von Viking Sonden kopiert
        \item Frei programmierbares System
        \item Aufteilung in drei redundante Hauptsubsysteme
            \begin{itemize}
                \item Communication and Command Subsystem (CCS)
                \item Attitude and Articulation Control System (AACS)
                \item Flight Data Subsystem (FDS)
            \end{itemize}
    \end{itemize}
\end{frame}

\begin{frame}
    \frametitle{Communication and Command Subsystem I}
    \begin{itemize}
        \item Steuerung der anderen Subsysteme und der Sonde
        \item Kommunikation mit der Erde
        \item Kopiert aus der Viking Sonde
        \item Anpassungen an Ein-/Ausgabe interface
        \item AACSIN zur Überwachung des AACS
    \end{itemize}
\end{frame}


\begin{frame}
    \frametitle{Communication and Command Subsystem II}
    \begin{itemize}
        \item Drei mögliche Betriebsarten
        \item Immer eingeschaltet
        \item Aktiv zu 3\%-4\% der Zeit
        \item 18-bit Prozessor mit 64 Befehlen
        \item Komplettes System redundant
    \end{itemize}
\end{frame}


\begin{frame}
    \frametitle{Communication and Command Subsystem III}
    \begin{center}
    \begin{figure}
		\includegraphics[height=6cm, width=11cm]{pics/voyager_ccs.jpg}
        \caption{Aufbau des Viking CCS\cite{nasaphoto}}
                \end{figure}
    \end{center}
\end{frame}

\begin{frame}
    \frametitle{Flight Data Subsystem}
    \begin{itemize}
        \item Sammlung und Verarbeitung von Daten
        \item Hohe Ein-/Ausgabegeschwindigkeiten notwending
        \item Speicher aus CMOS Bausteinen
    \end{itemize}
    \begin{figure}
    \begin{center}
	\includegraphics[height=3.7cm, width=8cm]{pics/voyager_fds.jpg}
    \caption{Voyagers FDS\cite{nasaphoto}}
    \end{center}
    \end{figure}
\end{frame}

\begin{frame}
    \frametitle{Attitude and Articulation Control System}
    \begin{itemize}
        \item Kontrolle und Ausrichtung der Sonde
        \item Umgebaute Version des CCS mit HYPACE
        \item Erstes aktives System
        \item Beide AACS nie Gleichzeitig aktiv
        \item Gyromodus und Sternenmodus
    \end{itemize}
\end{frame}

\begin{frame}
    \frametitle{Missionsablauf}
    \begin{itemize}
        \item Start am 20. August und 5. September 1977
        \item Probleme mit Voyager 2 nach dem Start
        \item Letzte Kurskorrekturen wurden durchgeführt
        \item Planung für Jupiter Vorbeiflug
    \end{itemize}
\end{frame}

\begin{frame}
    \frametitle{Jupitervorbeiflug}
    \begin{figure}
    \begin{tikzpicture}
        \node (img1) {
            \includegraphics[height=4.5cm]{pics/voyager_io.jpg}
        };
        \node (img2) at (img1.south east)[yshift=0.5cm,xshift=2cm] {\includegraphics[height=4cm]{pics/voyager_jupiter}};
        \end{tikzpicture}
        \caption{Links: Mond IO von Voyager 1, Rechts: Jupiter mit IO von Voyager 2\cite{nasaphoto}}
    \end{figure}
\end{frame}


%\begin{frame}
    %\frametitle{Swing-by Resultat}
    %\begin{figure} 
        %\includegraphics[height=6cm,width=7cm]{pics/s.png}
    %    \caption{Relative Geschwindigket von Voyager 2 zur Sonne}
    %\end{figure}

%\end{frame}

\begin{frame}
    \frametitle{Saturnvorbeiflug}
    \begin{figure} 
    \begin{tikzpicture}
        \node (img1) {
            \includegraphics[height=4.5cm]{pics/saturn_voyager1.jpg} 
        };
        \node (img2) at (img1.south east)[yshift=1.5cm,xshift=2cm] {\includegraphics[width=4cm]{pics/saturn_voyager2}};
        \end{tikzpicture}
        \caption{Links: Saturn von Voyager 1, Rechts: Saturnringe von Voyager 2\cite{nasaphoto}}
    \end{figure}
\end{frame}


\begin{frame}
    \frametitle{Voyager 1 - Missionsende}
    \begin{columns}
    \begin{column}[T]{0.5\textwidth}
    \begin{itemize}
        \item Keine weiteren \\ Planetenbesuche
        \item Als Testobjekt für Voyager 2 benutzt
        \item Seit 2007 Abschaltung der Systeme
        \item 2025-2030 Ende der Sonde
    \end{itemize}
    \end{column}
    \begin{column}[T]{0.5\textwidth}
        \begin{figure}
            \includegraphics[width=6cm, height=5cm]{pics/voyager1_end}
            \caption{Voyager 1 Flugbahn\cite{nasaphoto}}
        \end{figure}
    \end{column}
    \end{columns}
\end{frame}

\begin{frame}
    \frametitle{Uranusvorbeiflug}
    \begin{figure} 
    \begin{tikzpicture}
        \node (img1) {
            \includegraphics[height=4.5cm]{pics/uranus.jpg} 
        };
        \node (img2) at (img1.south east)[yshift=1.5cm,xshift=2cm] {\includegraphics[width=4cm]{pics/uranus_ringe.jpg}};
        \end{tikzpicture}
        \caption{Links: Uranus von Voyager 2, Rechts: Uranusringe von Voyager 2\cite{nasaphoto}}
    \end{figure}
\end{frame}

\begin{frame}
    \frametitle{Neptunvorbeiflug}
    \begin{figure}
    \begin{tikzpicture}
        \node (img1) {
            \includegraphics[height=4.5cm]{pics/neptun.jpg}
        };
        \node (img2) at (img1.south east)[yshift=1.5cm,xshift=2cm] {\includegraphics[width=4cm]{pics/triton.jpg}};
        \end{tikzpicture}
        \caption{Links: Neptunwolken von Voyager 2, Rechts: Mond Triton von Voyager 2\cite{nasaphoto}}
    \end{figure}
\end{frame}

\begin{frame}
    \frametitle{Voyager - Missionsende}
        \begin{figure}
            \includegraphics[width=8cm,height=6cm]{pics/voyager_traj.jpg}
            \caption{Voyager 1 und 2 Flugbahnen\cite{nasaphoto}}
        \end{figure}
\end{frame}
