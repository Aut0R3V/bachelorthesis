
\begin{frame}
	\frametitle{Galileo - Weg zum Start}
	\begin{itemize}
		\item Antrag genehmigt am 14.4.1977
		\item Erforschung von Jupiter und seinen Monden
		\item Flugbahn\"anderungen durch Monde
		\item Raumsonde und Atmosph\"arenprobe
		\item Geplanter Start 1982
	\end{itemize}
\end{frame}


\begin{frame}
	\frametitle{Verteiltes Rechnen auf Galileo}
	\begin{itemize}
		\item CDS - Command and Data Subsystem
		\item DPU - Data Processing Units
		\item AACS - Attitude and Articulation Control Subsystem
		\item Atmosph\"arensonde System
		\item Systemsynchronisation über Clock alle 2.5 Millisekunden
	\end{itemize}
\end{frame}

\begin{frame}
	\frametitle{CDS - Command and Data Subsystem}
	\begin{itemize}
		\item CDS besteht aus HLM, zwei LLM und Bus
		\item HLM - High-level module
		\begin{itemize}
			\item RCA 1802, Speicher, Bussystem
			\item Bus kontrolle, System\"uberwachung
		\end{itemize}
		\item LLM - Low-level modules
		\begin{itemize}
			\item RCA 1802, Speicher
			\item \"Uberwachung der Raumsonde, Atmosph\"arenprobe
		\end{itemize}
		\item Identisches Backup
	\end{itemize}
\end{frame}

\begin{frame}
	\frametitle{CDS Software}
	\begin{itemize}
		\item Assembler mit Macros (IF, DO, ASSIGN)
		\item Vordergrund Prozesse z.B. Selbstest, Bus Kontrolle
		\item Hintergrundprozesse 6 Virtuelle Maschinen
		\item Jede VM hat einen Programmstack im Speicher
		\item VM mit höherer Priorität bekommen mehr Rechenzeit
	\end{itemize}
\end{frame}

\begin{frame}
	\frametitle{DPU - Data Processing Unit}
	\begin{itemize}
		\item Eine DPU pro Experiment
		\item RCA 1802 und Speicher
		\item Antenne im Time-Sharing Betrieb
		\item Anfallende Daten gehen im n\"achsten Zeitslot zur Erde
		\item Assembler
		\item Eine DPU wurde mit FORTH programmiert
	\end{itemize}
\end{frame}

\begin{frame}
	\frametitle{AACS - Attitude and Articulation Control Subsystem}
	\begin{itemize}
		\item Basiert auf Voyager's System
		\item Zwei redundande ATAC - Applied Technologies Advanced Computer
		\item Orientierung der Sonde im Raum
		\item Ausrichtung der Instrumentenplattform
		\item Z\"undung der Triebwerke
	\end{itemize}
\end{frame}

\begin{frame}
	\frametitle{AACS}
	\begin{itemize}
		\item 16 Bit Computer bestehend aus vier 4 Bit AMD 2900 bitslice Prozessoren
		\item 9,5 MHz
		\item HAL/S Programme
		\item Assembler Betriebssystem GRACOS - Galileo Real-time Attitude Control Operating System
		\item SEU Problem anf\"allig
		\item Ohne Versp\"atung beim Start Ausfall wegen Io m\"oglich
	\end{itemize}
\end{frame}

\begin{frame}
	\frametitle{RCA 1802}
	\begin{itemize}
		\item Strahlengeh\"artet Silizium auf Saphir
		\item 3.2 MHz Taktrate
		\item RISC mit 8 Bit Befehlen, nur 16 Befehle
		\item 16 Register zu je 16 Bit
		\item Programmcounter und Indexregister kann frei gew\"ahlt werden
		\item I/O direkt vom und in den Speicher
	\end{itemize}
\end{frame}


\begin{frame}
\frametitle{Start}
	\begin{itemize}
		\item Start verschoben in 1982, 1985 und 1986
		\begin{itemize}
			\item Verz\"ogerungen beim Spaceshuttle Programm
			\item Pr\"asident Reagan
			\item Challenger Katastrophe
		\end{itemize}
		\item Folgen jeder Verschiebung
		\begin{itemize}
			\item Neue Route zum Jupiter
			\item Umbau der Raumsonde
		\end{itemize}
		\item Start am 18.10.1989
	\end{itemize}
\end{frame}

\begin{frame}
\frametitle{Flug zum Jupiter \"Ubersicht}
	\begin{itemize}
		\item Dauer 6 Jahre
		\item Schwung holen an Venus und zwei mal an Erde
		\item Vorbeiflug an Planetoiden Gaspra und Ida
		\item Primärmission beginnt beim Jupiter
	\end{itemize}
\end{frame}


\begin{frame}
	\begin{figure}
		\centering
		\includegraphics[width=0.5\textwidth]{pics/venus.jpg}
		\caption{Vorbeiflug an Venus am 14.2.1990\cite{nasaphoto}}
		\label{VENUS}
	\end{figure}
\end{frame}

\begin{frame}
	\begin{figure}
		\centering
		\includegraphics[width=0.5\textwidth]{pics/erde0.jpg}
		\caption{Vorbeiflug an Erde am 12.12.1990\cite{nasaphoto}}
		\label{ERDE0}
	\end{figure}
\end{frame}

\begin{frame}
	\begin{figure}
		\centering
		\includegraphics[width=0.5\textwidth]{pics/gaspra.jpg}
		\caption{Vorbeiflug an Planetoiden Gaspra am 29.10.1991\cite{nasaphoto}}
		\label{GASPRA}
	\end{figure}
\end{frame}

\begin{frame}
	\begin{figure}
		\centering
		\includegraphics[width=0.7\textwidth]{pics/ida.jpg}
		\caption{Vorbeiflug an Planetoiden Ida und sein Mond Dactyl am 28.8.1991\cite{nasaphoto}}
		\label{IDA}
	\end{figure}
\end{frame}

\begin{frame}
	\begin{figure}
		\centering
		\includegraphics[width=0.7\textwidth]{pics/jupiter.jpg}
		\caption{Komet Shoemaker-Levy 9 schl\"agt auf Jupiter ein. 22.7.1994\cite{nasaphoto}}
		\label{JUPITER}
	\end{figure}
\end{frame}

\begin{frame}
	\begin{figure}
		\centering
		\includegraphics[width=0.8\textwidth]{pics/monde.jpg}
		\caption{Die Galileischen Monde Io, Europa, Ganymede, und Callisto\cite{nasaphoto}}
		\label{GALILEO_MONDE}
	\end{figure}
\end{frame}

\begin{frame}
	\frametitle{Probleme wärend der Reise zum Jupiter}
	\begin{itemize}
		\item HGA - Hochgewinnantenne Entfaltet sich nicht
		\item NGA - Niedriggewinnantenne nur \"Ubertragungsrate 10 Bit/s
		
		\item Bandrekorder reagiert für 18 Stunden nicht
		\item Leichter Schaden f\"uhrt zu 16\% weniger Speicher
	\end{itemize}
\end{frame}

\begin{frame}
	\frametitle{Probleml\"osung durch Software Updates}
	\begin{itemize}
		\item Antenne Paketbasierter Betrieb
		\item DPU's erstellen und w\"ahlen Pakete
		\item Bilder werden auf Bandrekorder zwischengespeichert
		\item CDS bearbeitet und komprimiert Bilder
		\item DCT - Diskrete Cosinustransformation zur verlustbehaftete Bildkomprimierung (JPEG)
	\end{itemize}
\end{frame}

\begin{frame}
\frametitle{Prim\"armission}
	\begin{itemize}
		\item 7.5.1995 bis 7.12.1997
		\item 11 Orbits um Jupiter
		\item Relativ naher Vorbeiflug an je einem Mond
		\item Unerwartet große Treibstoffreserven nach Mission
	\end{itemize}
\end{frame}

\begin{frame}
	\frametitle{Verl\"angerungen}
	\begin{itemize}
		\item GEM - Galileo Europa Mission
		\begin{itemize}
			\item 8.12.1997 bis 31.12.1999
			\item 14 Orbits, 7 Vorbeifl\"uge an Europa
		\end{itemize}
		\item GMM - Galileo Millennium Mission
		\begin{itemize}
			\item 1.2.2000 bis 21.9.2003
			\item Probleme h\"aufen sich
			\item Riskante Vorbeifl\"uge an Io
		\end{itemize}
	\end{itemize}
\end{frame}

\begin{frame}
	\frametitle{Vielen Dank f\"ur Ihre Aufmerksamkeit!}
	\begin{figure}
		\centering
		\includegraphics[width=0.6\textwidth]{pics/number_of_computers.png}
		\caption{xkcd Comic\cite{xkcd}}
		\label{NUMBER_OF_COMPUTERS}
	\end{figure}
\end{frame}
